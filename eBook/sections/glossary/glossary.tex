
\newglossaryentry{Determinism}{
  name=Determinism,
  description={Transaction and blockchain behavior are predictable, given a sort of input and outputs, once the fee is decided the transaction hash will always be the same}
}
\newglossaryentry{Composability}{
  name=Composability,
  description={Also referred to as transaction in transaction, it's the ability to interact with multiple parties in the same transaction, this is not possible in the account model, however, this also raises the concurrency issue when two parties or transactions want to spend the same utxo}
}
\newglossaryentry{Liquid Staking}{
  name=Liquid Staking,
  description={Cardano staking is referred to as Liquid, no locking mechanism is needed to get the staking rewards. This becomes useful because users can move their ADA around inside smart contracts while keeping the delegation rewards }
}
\newglossaryentry{orderbook}{
  name=orderbook,
  description={In this configuration each order placed by users is a single entry with a price and amount of token willing to sell (ADA or native tokens), swaps happen matching the orders }
}
\newglossaryentry{AMM}{
  name=AMM,
  description={Automatic market maker dexes involve a liquidity pool, the pool has two pair tokens, usually ADA and the Cardano native token, users can sell or buy tokens from this pool and the price is adjusted according to the market need }
}

\newglossaryentry{CIP}{
  name=CIP,
  description={Cardano Improvement Proposals that if approved can change the current ledger or chain parameters, usually are also standards to develop in a similar way between projects  }
}

\newglossaryentry{inputs}{
    name=inputs,
    description={Inputs in a UTXO model transaction specify which unspent outputs are being consumed, so which funds coming from previous transactions are being spent. }
}

\newglossaryentry{epoch}{
    name=epoch,
    description={An epoch in Cardano is a fixed period during which a set of blocks is produced. The duration of an epoch is predefined and consistent. As of the current Cardano implementation, an epoch lasts for 5 days. At the end of each epoch, rewards are calculated and distributed, and a new epoch begins. Epochs help structure the blockchain into manageable time periods, enabling efficient consensus and reward mechanisms.
    }
}

\newglossaryentry{block}{
    name=block,
    description={A block in Cardano is a record of transactions and other information produced by a slot leader during a slot. Blocks are added to the blockchain sequentially. Each block contains a header with metadata, such as the previous block hash, and a body that includes the transaction data and other relevant information. Blocks are produced by slot leaders, which are chosen through the Ouroboros consensus protocol, Cardano's proof-of-stake mechanism. Blocks are essential for maintaining the integrity and continuity of the blockchain, as they confirm and validate transactions.
    }
}


\newglossaryentry{slot}{
    name=slot,
    description={A slot is a smaller time unit within an epoch. An epoch is divided into a large number of slots. Each slot represents a potential opportunity to produce a block. In the current implementation of Cardano, there are 432,000 slots in an epoch, with each slot lasting 1 second. However, not every slot will necessarily have a block produced, as block production depends on the consensus protocol and slot leader election.
    }
}